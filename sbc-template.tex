%%%%
%       _  _____  ___  _____  _   _  ___ 
%      / \|_   _||_ _||_   _|| | | |/ __|
%     / △ \ | |   | |   | |  | |_| |\__ \
%    /_/¯\_\|_|  |___|  |_|   \___/ |___/
%                 EDUCAÇÃO
%    Modelo de TCC - Ciência da Computação
%%%%

\documentclass[12pt]{article}

\usepackage{sbc-template}
\usepackage{graphicx, url}
\usepackage[utf8x]{inputenc}
\usepackage[brazil]{babel}
\usepackage[T1]{fontenc}
\usepackage[english=american]{csquotes}
\usepackage{float}
\usepackage{comment}
\usepackage{amsmath}
\usepackage{amssymb}
\usepackage{enumitem}
\usepackage{subcaption}
\usepackage{setspace}
\usepackage{listings}
\usepackage[scaled=0.85]{beramono}
\usepackage{tabularray}
\usepackage[hidelinks, breaklinks=true]{hyperref}
%\usepackage{hyperref}
%\hypersetup{
%    colorlinks=true,
%    allcolors=blue,
%}

% %%%%%%%%%%%%%%%%%%%%%%%%%%%%%%%%%%%%%%%%%%%%%%%%%%%%%%%%%%%%%%
% define o modelo de referencias
\usepackage[style=abnt, giveninits, uniquename=mininit, repeatfields]{biblatex}

% indica o arquivo com as referencis bibliograficas
\addbibresource{sbc-template.bib}

% carrega o pacote com alterações para Computação Atitus
\usepackage{sty/cc_atitus}
% %%%%%%%%%%%%%%%%%%%%%%%%%%%%%%%%%%%%%%%%%%%%%%%%%%%%%%%%%%%%%%

% %%%%%%%%%%%%%%%%%%%%%%%%%%%%%%%%%%%%%%%%%%%%%%%%%%%%%%%%%%%%%%
% REFERENCIAS DEVERÃO SER INCLUÍDAS NO ARQUIVO: sbc-template.bib
%
% SOBRE CITAÇÕES:
% Para citar no padrão '(Autores, ano)' use: \cite{chave}
% Para citar no padrão 'Autores (ano)'  use: \textcite{chave}
% Para citar no padrão 'Autores'        use: \citelastname{chave}
% Para citar no padrão 'Autor (ano), Autor (ano) e Autor (ano)' use: \textcite{chave1,chave2,chave3}
% Para citar no padrão '(Autor, 2009c; Outro Autor, 2009; Outro Autor, 2015)' use: \cite{chave1, chave2, chave3}

% Demais exemplos ver documento:
%  https://github.com/abntex/biblatex-abnt/raw/master/doc/biblatex-abnt.pdf
%
% Normas ABNT: https://usp.br/sddarquivos/arquivos/citacoes10520.pdf
%              https://usp.br/sddarquivos/arquivos/abnt6023.pdf
%              https://www.fct.unesp.br/Home/Biblioteca/abnt/abnt-atualizado-dez-2023.pdf
%
% %%%%%%%%%%%%%%%%%%%%%%%%%%%%%%%%%%%%%%%%%%%%%%%%%%%%%%%%%%%%%%
% Atualizações no modelo
% 2024-09-23: (fahadkalil) Ajuste para quebra automática de linhas nas células de uma tabela
% 2024-10-17: (fahadkalil) Inserção de comentários com outras citações e indicação de uso do comando "\enquote"
% 2025-06-05: (fahadkalil) Adição de novos exemplos de referências bibliográficas, figuras e tabelas / Correção para que campos desnecessários no arquivo .bib sejam ignorados na renderização / Correção de espaçamento no comando \begin{itemize} usando opções adicionais na seção Trabalhos Relacionados / Correção de recuos quando utilizado o comando \begin{description} / Correção na fonte e espaçamentos do comando 'lstlisting'
% %%%%%%%%%%%%%%%%%%%%%%%%%%%%%%%%%%%%%%%%%%%%%%%%%%%%%%%%%%%%%%
% CABEÇALHO
\title{Título do Trabalho} % titulo

\author{Nome aluno\inst{1}, Nome orientador\inst{1}} % autor principal, orientador

\address{Ciência da Computação -- Atitus Educação\\
Passo Fundo -- RS -- Brasil
\email{aluno@atitus.edu.br, orientador@atitus.edu.br}
}
% %%%%%%%%%%%%%%%%%%%%%%%%%%%%%%%%%%%%%%%%%%%%%%%%%%%%%%%%%%%%%%

\begin{document}
\maketitle % Não remova essa linha!

\begin{abstract} % resumo em inglês
  Escreva seu resumo em língua estrangeira (inglês)...
\end{abstract}
     
\begin{resumo}
  Resumo do trabalho (português)...  
\end{resumo}

\section{Introdução}
Cenários, motivação, justificativa, problema de pesquisa, objetivo geral e específicos (taxonomia de Bloom);

\section{Referencial Teórico}
Voltado ao objetivo geral (teoria por trás do método), deve conter os assuntos-base da pesquisa, fazendo citações indiretas e diretas curtas.

\begin{comment}
testando comentario que não aparecerá no PDF
\end{comment}

\subsection{Exemplos de citação indireta}

Segundo \textcite{spinello2024}, o trabalho de conclusão deve ter citações retiradas de artigos científicos encontrados nas bases de dados. 

O trabalho de conclusão deve ter citações retiradas de artigos científicos encontrados nas bases de dados \cite{spinello2024}.

\subsection{Exemplos de citação direta curta}

Segundo \textcite{spinello2024} \enquote{o trabalho de conclusão deve ter citações retiradas de artigos científicos encontrados nas bases de dados}. Note que para colocar um texto entre aspas, usamos o comando \verb|\enquote{texto}|.

Ressalta-se que o \enquote{trabalho de conclusão deve ter citações retiradas de artigos científicos encontrados nas bases de dados} \cite{badgujar2024}.

No estudo comparativo apresentado em \textcite[p. 107]{rabello2010} ...

No trabalho de \textcite{souza2020} ...

No artigo de \textcite{estevao2023} ...

Nos trabalhos de \textcite{badgujar2024, estevao2023} são aplicadas técnicas de ...

\textsc{Citação direta longa devem ser evitadas em artigos científicos!}

\section{Trabalhos Relacionados}

Trabalhos semelhantes aos objetivos específicos, sempre detalhando ao final da seção a diferença com o trabalho proposto (quantidade -- 5 trabalhos);

Neste item serão apresentados os principais trabalhos que possuem uma relação com o assunto definido neste estudo....
% Itens com marcadores
\begin{itemize}
     \item \textbf{Título do artigo 01 \cite{ogliari2019}}
     
     Primeiro parágrafo indicar uma introdução do assunto...
     
     No segundo: o que o estudo procurou analisar, qual o objetivo...
     
     No terceiro: o que foi desenvolvido, qual aplicação/experimento foi realizado...
     
     Último: em quais conclusões o trabalho chegou
        
    \item \textbf{Título do artigo 02 (Autor, ano)}
    
     Primeiro parágrafo indicar uma introdução do assunto...
     
     No segundo: o que o estudo procurou analisar, qual o objetivo...
     
     No terceiro: o que foi desenvolvido, qual aplicação/experimento foi realizado...
     
     Último: em quais conclusões o trabalho chegou...
    
\end{itemize}

\section{Materiais e Métodos}
    Tecnologias, instrumentos e procedimentos que serão usados no estudo. O Algoritmo~\ref{alg:id_algo} se refere ao método de ordenação \texttt{Bubblesort} expresso em linguagem \texttt{Python}.

    Uma lista numérica:
    \begin{enumerate}
        \item Item 1
        \item Item 2
    \end{enumerate}

    Uma lista definida com letras sequenciais:
    \begin{enumerate}[label=\alph*)]
        \item Item 1
        \item Item 2
    \end{enumerate}    

    % Código-fonte formatado
    % Ver: https://en.wikibooks.org/wiki/LaTeX/Source_Code_Listings
    \begin{lstlisting}[  
        %float,
        language=Python, 
        frame=single, 
        numbers=left,
        caption={Método de ordenação Bubblesort},
        label={alg:id_algo} % id para referenciar
        ]        
def bubble_sort(alist):
    for i in range(len(alist)-1,0,-1):
        for j in range(i):
            if alist[i]>alist[i+1]:
                temp = alist[i]
                alist[i] = alist[i+1]
                alist[i+1] = temp    
    \end{lstlisting}

\section{Resultados e Discussão}
Essa seção deverá ser escrita na segunda parte do trabalho, conhecida como TCC2, e deverá conter os resultados dos experimentos realizados, discussão comparando os resultados obtidos com outros encontrados em trabalhos similares, além de um parágrafo apontando as limitações da metodologia adotada.

 % Figura
    % https://pt.overleaf.com/learn/latex/Inserting_Images
    \begin{figure}[h!] % h! pede para o compilador tentar incluir a figura na posição em que foi declarada;
        \centering
        \includegraphics[width=0.4\textwidth]{fig1.jpg} % altere a width (intervalo entre 0.0 e 1), caso a imagem não fique na posição correta
        \caption{Exemplo de uso de figura}
        \label{fig:figura1}
    \end{figure}

    % Figuras (usando subfigure)
    % https://latex-tutorial.com/subfigure-latex/
    \begin{figure}[h!] % h! pede para o compilador tentar incluir a figura na posição em que foi declarada;
        \centering
        \caption{Exemplo com sub-figuras}
        \begin{subfigure}{0.4\textwidth}
            \includegraphics[width=\textwidth]{fig1.jpg}
            \caption{Sub-figura 1}
            \label{fig:sub_fig1}
        \end{subfigure}
        \hspace{1em}
        \begin{subfigure}{0.4\textwidth}
            \includegraphics[width=\textwidth]{fig1.jpg}
            \caption{Sub-figura 2}
            \label{fig:sub_fig2}
        \end{subfigure}
        \label{fig:figura2}
    \end{figure}
    
    % Tabelas
    % Editor online: https://www.latex-tables.com
    % Referência: https://ctan.math.illinois.edu/macros/latex/contrib/tabularray/tabularray.pdf
     \begin{table}[H] % h! pede para o compilador tentar incluir a tabela na posição em que foi declarada;
        \centering
        \caption{Minha tabela}
        \label{tab:tabela1}
        \begin{tblr}{
          colspec = {X[l] X[l]}, % Tipo X define células com quebra automática de linha. Use: l=left, r=right, c=center ou j=justify para alinhamento dentro das células
          width = \linewidth,
          hlines, % inclui borda horizontal
          vlines, % inclui borda vertical        
        }
        \textbf{cabeçalho 1} & \textbf{cabeçalho 2} \\
        {texto à esquerda} & {Existem muitas variações das passagens do Lorem Ipsum disponíveis, mas a maior parte sofreu alterações de alguma forma.}
        \end{tblr}        
    \end{table}

    % tabela com subtabelas
    \begin{table}[H]
        \centering
        \caption{Tabela com sub-tabelas}
        \label{tab:tabela2}
        \begin{subtable}{0.45\linewidth} %subtabela
            \centering            
            \begin{tblr}{
              colspec = {X[l] X[l]}, % Tipo X define células com quebra automática de linha. Use: l=left, r=right, c=center ou j=justify para alinhamento dentro das células
              width = \linewidth,
              hlines, % inclui borda horizontal
              vlines, % inclui borda vertical        
            }
            \textbf{cabeçalho 1} & \textbf{cabeçalho 2} \\
            {Contrary to popular belief, Lorem Ipsum is not simply random text. It has roots in a piece of classical Latin literature from 45 BC, making it over 2000 years old.} & {The standard chunk of Lorem Ipsum used since the 1500s is reproduced below for those interested.}
            \end{tblr}
            \caption{Sub-tabela da esquerda}
        \end{subtable}
        \hfill
        \begin{subtable}{0.45\linewidth} %subtabela
            \centering            
            \begin{tblr}{
                  colspec = {X[l] X[l]}, % Tipo X define células com quebra automática de linha. Use: l=left, r=right, c=center ou j=justify para alinhamento dentro das células
                  width = \linewidth,
                  hlines, % inclui borda horizontal
                  vlines, % inclui borda vertical        
                }
                \textbf{cabeçalho 1} & \textbf{cabeçalho 2} \\
                {Contrary to popular belief, Lorem Ipsum is not simply random text. It has roots in a piece of classical Latin literature from 45 BC, making it over 2000 years old.} & {The standard chunk of Lorem Ipsum used since the 1500s is reproduced below for those interested.}
            \end{tblr}
            \caption{Sub-tabela da direita}
        \end{subtable}
    \end{table}

\section{Considerações Finais}
Essa seção deverá ser escrita na segunda parte do trabalho, conhecida como TCC2.

% %%%%%%%%%%%%%%%%%%%%%%%%%%%%%%%%%%%%%%%%%%%%%%%%%%%%%%%%%%%%%%
% Seção de Referências (gerada automaticamente)
\printbibliography  % Não remover esta linha
% %%%%%%%%%%%%%%%%%%%%%%%%%%%%%%%%%%%%%%%%%%%%%%%%%%%%%%%%%%%%%%

\end{document}